\chapter{RL Framework}

强化学习框架代码

关键类:
\begin{itemize}
\item[-]
环境 {\bf env}

\item[-]
模型 {\bf model}

\item[-]
算法 {\bf algorithm}

\item[-]
agent {\bf agent}

\item[-]
回放内存 {\bf rpm}
\end{itemize}


关键技术:
\begin{itemize}
\item
方法的多态:如同调用父类中定义的 virtual 方法,子类中有不同的实现。
	\begin{itemize}
	\item
	python 中不具备类似的类的 virtual 方法

	\item
	可将不同的对象放入同一个列表,遍历列表,调用对象的相同方法名的函数。
	这种方式甚至可以调用不同类但具有相同方法名的对象(这些不同类可以无相同父类)

	\item
	可以把父类的子类的对象封装进一个链表,对父类 virtual 方法的调用对应为对链表中对象的同名方法的调用。
	\end{itemize}

\item
Assign functions to a variable:
\begin{lstlisting}[language=Python]
# function defined
def multiply_num(a):
	b = 40
	r = a*b
	return r


# drivercode
# assigning function
z = multiply_num

# invoke function
print(z(6))
print(z(10))
print(z(100))
\end{lstlisting}

\end{itemize}


\section{TD3 vs ycl codes review}

训练的入口:main.py ln136
\begin{lstlisting}[language=Python]
		# Train agent after collecting sufficient data
		if t >= args.start_timesteps:
			policy.train(replay_buffer, args.batch_size)
\end{lstlisting}

ycl 对应的训练入口:TD3/train.py ln23
\begin{lstlisting}[language=Python]
# Run episode for training
def run_train_episode(agent, env, rpm):
    action_dim = env.action_space.shape[0]
    obs = env.reset()
    done = False
    episode_reward = 0
    episode_steps = 0
    while not done:
        episode_steps += 1
        # Select action randomly or according to policy
        if rpm.size() < WARMUP_STEPS:
            action = np.random.uniform(-1, 1, size=action_dim)
        else:
            action = agent.sample(obs)

        # Perform action
        next_obs, reward, done, _ = env.step(action)
        terminal = float(done) if episode_steps < env._max_episode_steps 
                               else 0

        # Store data in replay memory
        if reward <= -100:
            reward = -1#100
            rpm.append(obs, action, reward, next_obs, 1)
            # replay_buffer.add(s, a, r, s_prime, True)
        else:
            rpm.append(obs, action, reward, next_obs, 0)
        # rpm.append(obs, action, reward, next_obs, terminal)

        obs = next_obs
        episode_reward += reward

        # Train agent after collecting sufficient data
        if rpm.size() >= WARMUP_STEPS:
            batch_obs, batch_action, batch_reward, batch_next_obs, 
                batch_terminal = rpm.sample_batch(BATCH_SIZE)
            agent.learn(batch_obs, batch_action, batch_reward, 
                batch_next_obs, batch_terminal)

    return episode_reward, episode_steps
\end{lstlisting}

上面第37行的 \textcolor{magenta}{agent.learn} 是训练的入口。也就是说 TD3 将训练过程封装在 
policy 中,而 ycl 将训练过程封装在 agent 中。孰优孰劣后面再继续加以分析。

\subsection{TD3 代码中初始化时(main.py 中)产生的关键对象}

\begin{itemize}
\item[-]
环境 {\bf env}: ln66
\begin{lstlisting}[language=Python]
	env = gym.make(args.env)
\end{lstlisting}

\item[-]
策略(算法) {\bf policy(algorithm)}: ln86
\begin{lstlisting}[language=Python]
	# Initialize policy
	if args.policy == "TD3":
		# Target policy smoothing is scaled wrt the action scale
		kwargs["policy_noise"] = args.policy_noise * max_action
		kwargs["noise_clip"] = args.noise_clip * max_action
		kwargs["policy_freq"] = args.policy_freq
		policy = TD3.TD3(**kwargs)
	elif args.policy == "OurDDPG":
		policy = OurDDPG.DDPG(**kwargs)
	elif args.policy == "DDPG":
		policy = DDPG.DDPG(**kwargs)
\end{lstlisting}

\item[-]
模型 {\bf model}: ln98
\begin{lstlisting}[language=Python]
	if args.load_model != "":
		policy_file = file_name if args.load_model == "default" 
            else args.load_model
		policy.load(f"./models/{policy_file}")
\end{lstlisting}

\item[-]
回放内存 {\bf rpm}: ln102
\begin{lstlisting}[language=Python]
	replay_buffer = utils.ReplayBuffer(state_dim, action_dim)
\end{lstlisting}
\end{itemize}


\subsection{ycl 代码中初始化时(tain.py 中)产生的关键对象}

\begin{itemize}
\item[-]
环境 {\bf env}:ln85
\begin{lstlisting}[language=Python]
    env = gym.make(args.env)
\end{lstlisting}

\item[-]
模型 {\bf model}:ln96
\begin{lstlisting}[language=Python]
    model = Model(obs_dim, action_dim)
\end{lstlisting}

\item[-]
算法 {\bf algorithm}:ln97
\begin{lstlisting}[language=Python]
    algorithm = TD3(
        model,
        gamma=GAMMA,
        tau=TAU,
        actor_lr=ACTOR_LR,
        critic_lr=CRITIC_LR,
        policy_noise=EXPL_NOISE,  # Noise added to target policy during critic update
        noise_clip=CLIP_NOISE,
        policy_freq=args.policy_freq)
\end{lstlisting}

\item[-]
agent {\bf agent}:ln106
\begin{lstlisting}[language=Python]
    agent = Agent(algorithm, act_dim=action_dim, expl_noise=EXPL_NOISE)
\end{lstlisting}

\item[-]
回放内存 {\bf rpm}:ln107
\begin{lstlisting}[language=Python]
    rpm = ReplayMemory(
        max_size=MEMORY_SIZE, obs_dim=obs_dim, act_dim=action_dim,seed=args.seed)
\end{lstlisting}

\end{itemize}


\subsection{The encapsulation of class TD3}

Agent设计为能在同一场景下选择多种策略算法。




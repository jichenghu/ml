% A size substituation of 0.5pt is harmless, but if you want to avoid it load the fix-cm package:
\RequirePackage{fix-cm}
\RequirePackage[hyphens]{url}
\documentclass{book}
%\let\Tiny=\tiny
\usepackage[UTF8]{ctex}
\usepackage{import}

%\PassOptionsToPackage{hyphens}{url}\usepackage{hyperref}

% https://texample.net/tikz/examples/feature/trees/
\usepackage{tikz}
\usetikzlibrary{arrows}

% https://texample.net/tikz/examples/red-black-tree/
\tikzset{
  treenode/.style = {align=left, %center, 
  inner sep=0pt, text centered,
    font=\sffamily},
  arn_n/.style = {treenode, circle, white, font=\sffamily\bfseries, draw=black,
    fill=black, text width=1.0em},% arbre rouge noir, noeud noir
  arn_r/.style = {treenode, circle, black, draw=black, 
    text width=1.5em, very thick},% arbre rouge noir, noeud rouge
  arn_x/.style = {treenode, rectangle, draw=black,
    minimum width=0.5em, minimum height=0.5em}% arbre rouge noir, nil
}

%\usepackage{amsmath}
%\usepackage{lmodern}
\usepackage{amsfonts, amsmath, amssymb, amsthm, array, bm, breakurl, cite, color, 
fancyhdr, float, framed, geometry, graphicx, hyperref, listings, 
mathrsfs, mathtools, setspace, tcolorbox, url, xcolor}
%\usepackage[keeplastbox]{flushend}
%\fontsize{11pt}{12pt}\selectfont

\usepackage[thicklines]{cancel}
% 可以设置线条颜色,默认是黑色
%\usepackage{xcolor}
\renewcommand{\CancelColor}{\color{red}}

\usepackage{caption}
\usepackage{subcaption}
\usepackage{subfigure}

\lstset{numbers=left, %设置行号位置
        numberstyle=\tiny, %设置行号大小
        keywordstyle=\color{blue}, %设置关键字颜色
        commentstyle=\color[cmyk]{1,0,1,0}, %设置注释颜色
        frame=single, %设置边框格式
        escapeinside=``, %逃逸字符(1左面的键),用于显示中文
        %breaklines, %自动折行
        extendedchars=false, %解决代码跨页时,章节标题,页眉等汉字不显示的问题
        xleftmargin=2em,xrightmargin=2em, aboveskip=1em, %设置边距
        tabsize=4, %设置tab空格数
        showspaces=false %不显示空格
       }

% hyperref setup
\hypersetup{
    colorlinks=true,
    linkcolor=blue,
    filecolor=magenta, 
    urlcolor=cyan,
    pdftitle={Overleaf Example},
    pdfpagemode=FullScreen,}

%\includeonly{entropy}

\usepackage[
backend=biber,
style=alphabetic,
sorting=ynt
]{biblatex}
\addbibresource{rl.bib}

%\usetheme{Madrid}

\geometry{left=1.5cm, right=1.2cm, top=2.5cm, bottom=2.0cm}
\linespread{0.1}

% pdflatex, xelatex

\title{REINFORCEMENT LEARNING}
\author{Jicheng Hu}
\date{\today}

\linespread{1.5}
\definecolor{shadecolor}{RGB}{231, 231, 255}
\newenvironment{emp_box}{\begin{shaded}\par\noindent}{\end{shaded}\par\noindent}
\newcounter{problemname}
\newcounter{examplename}
\newenvironment{problem}{\begin{shaded}\stepcounter{problemname}\par\noindent\textbf{题目\arabic{problemname}. }}{\end{shaded}\par}
\newenvironment{example}{\begin{shaded}\stepcounter{examplename}\par\noindent\textbf{例\arabic{examplename}. }}{\end{shaded}\par}
\newenvironment{solution}{\par\noindent\textbf{解答. }}{\par}
\newenvironment{note}{\par\noindent\textbf{题目\arabic{problemname}的注记. }}{\par}

\newtheorem{definition}{Definition}[section]     % 跟随⼆级标题排序
\newtheorem{theorem}{Theorem}[subsection]        % 跟随三级标题排序
\newtheorem{lemma}[theorem]{Lemma}
\newtheorem{corollary}[theorem]{Corollary}

%\renewcommand{\headrulewidth}{1pt} %页眉线宽,设为0可以去页眉线

\newdimen\doublelineskip % 两横线间的距离
\setlength\doublelineskip{2pt}

\setlength\headheight{21pt}

\fancypagestyle{headings}{
  \renewcommand\headrulewidth{0.4pt}
}
\fancypagestyle{doubleline}[headings]{
  \renewcommand\headrule{%
    \hrule height\headrulewidth width\headwidth%
    \vskip \doublelineskip%
    \hrule height\headrulewidth width\headwidth}
}

%\pagestyle{headings}

% usepackage{pythonhighlight}

%\makeatletter
%\g@addto@macro{\UrlBreaks}{\UrlOrds}
%\makeatother

\def\UrlBreaks{\do\A\do\B\do\C\do\D\do\E\do\F\do\G\do\H\do\I\do\J
\do\K\do\L\do\M\do\N\do\O\do\P\do\Q\do\R\do\S\do\T\do\U\do\V
\do\W\do\X\do\Y\do\Z\do\[\do\\\do\]\do\^\do\_\do\`\do\a\do\b
\do\c\do\d\do\e\do\f\do\g\do\h\do\i\do\j\do\k\do\l\do\m\do\n
\do\o\do\p\do\q\do\r\do\s\do\t\do\u\do\v\do\w\do\x\do\y\do\z
\do\.\do\@\do\\\do\/\do\!\do\_\do\|\do\;\do\>\do\]\do\)\do\,
\do\?\do\'\do+\do\=\do\#}


\begin{document}

\frontmatter	% 前言部分,页码为小写罗马字母格式;其后的\chapter 不编号。

%\setcounter{secnumdepth}{3}		%增加编号深度
%\setcounter{tocdepth}{3}		%增加目录深度

\tableofcontents

\thispagestyle{empty}
\mainmatter		% 表示开始正文部分的内容 使用数字进行页面编号,对其中的chapter进行编号(第一章、第二章……)

\begin{spacing}{1.0}    % 行间距变为single-space

\setcounter{page}{1}

%\import{D:/mygit/ml/reinforcementLearning/tex/}{front.tex}

\def\allfiles{}

% basic knowledge
\ifx\allfiles\undefined
\else
\import{D:/mygit/ml/reinforcementLearning/tex/}{rl_basic.tex}
\fi

% q-learning
%%%%%%%%%%%%%%%%%%%%%%%%%%%%%%
\ifx\allfiles\undefined
\else
\import{D:/mygit/ml/reinforcementLearning/tex/}{q_learning.tex}
\fi

% Markov Decision Processes
\ifx\allfiles\undefined
\else
\import{D:/mygit/ml/reinforcementLearning/tex/}{mdp.tex}
\fi

% policy
\ifx\allfiles\undefined
\else
\import{D:/mygit/ml/reinforcementLearning/tex/}{policy.tex}
\fi

% variational inference
\ifx\allfiles\undefined
\else
\import{D:/mygit/ml/reinforcementLearning/tex/}{variational_inference.tex}
\fi

% entropy
\ifx\allfiles\undefined
\else
\import{D:/mygit/ml/reinforcementLearning/tex/}{entropy.tex}
\fi

% td3
\ifx\allfiles\undefined
\else
\import{D:/mygit/ml/reinforcementLearning/tex/}{td3.tex}
\fi

% SAC
\ifx\allfiles\undefined
\else
\import{D:/mygit/ml/reinforcementLearning/tex/}{sac.tex}
\fi

% 逆强化学习
\ifx\allfiles\undefined
\else
\import{D:/mygit/ml/reinforcementLearning/tex/}{inverse_rl.tex}
\fi

% homotopy group
\ifx\allfiles\undefined
\else
\import{D:/mygit/ml/reinforcementLearning/tex/}{homotopy.tex}
\fi

% 框架代码
\ifx\allfiles\undefined
\else
\import{D:/mygit/ml/reinforcementLearning/tex/}{framework.tex}
\fi

% topics
\ifx\allfiles\undefined
\else
\import{D:/mygit/ml/reinforcementLearning/tex/topics/}{functionApproximator.tex}
\import{D:/mygit/ml/reinforcementLearning/tex/topics/}{clarify.tex}
\fi

% 附录
\ifx\allfiles\undefined
\else
\chapter{附录}
% 附件内容:最大熵
\import{D:/mygit/ml/reinforcementLearning/tex/}{entropy_max.tex}
% 附件内容:dynamic programming
\import{D:/mygit/ml/reinforcementLearning/tex/}{dp.tex}
% 附件内容:Temporal-Difference
\import{D:/mygit/ml/reinforcementLearning/tex/}{td.tex}
\fi

%\import{D:/mygit/ml/reinforcementLearning/tex/}{back.tex}

\printbibliography

\end{document}

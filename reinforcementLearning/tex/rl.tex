% A size substituation of 0.5pt is harmless, but if you want to avoid it load the fix-cm package:
\RequirePackage{fix-cm}
\documentclass{book}
%\let\Tiny=\tiny
\usepackage[UTF8]{ctex}
\usepackage{import}

%\usepackage{amsmath}
%\usepackage{lmodern}
\usepackage{amsfonts, amsmath, amssymb, amsthm, array, bm, cite, color, 
fancyhdr, float, framed, geometry, graphicx, hyperref, listings, 
mathrsfs, setspace, tcolorbox, xcolor}
%\usepackage[keeplastbox]{flushend}
%\fontsize{11pt}{12pt}\selectfont

\lstset{numbers=left, %设置行号位置
        numberstyle=\tiny, %设置行号大小
        keywordstyle=\color{blue}, %设置关键字颜色
        commentstyle=\color[cmyk]{1,0,1,0}, %设置注释颜色
        frame=single, %设置边框格式
        escapeinside=``, %逃逸字符(1左面的键),用于显示中文
        %breaklines, %自动折行
        extendedchars=false, %解决代码跨页时,章节标题,页眉等汉字不显示的问题
        xleftmargin=2em,xrightmargin=2em, aboveskip=1em, %设置边距
        tabsize=4, %设置tab空格数
        showspaces=false %不显示空格
       }

% hyperref setup
\hypersetup{
    colorlinks=true,
    linkcolor=blue,
    filecolor=magenta, 
    urlcolor=cyan,
    pdftitle={Overleaf Example},
    pdfpagemode=FullScreen,}

%\includeonly{entropy}

\usepackage[
backend=biber,
style=alphabetic,
sorting=ynt
]{biblatex}
\addbibresource{rl.bib}

%\usetheme{Madrid}

\geometry{left=1.5cm, right=1.2cm, top=2.5cm, bottom=2.0cm}
\linespread{0.1}

% pdflatex, xelatex

\title{REINFORCEMENT LEARNING}
\author{Jicheng Hu}
\date{\today}

\linespread{1.5}
\definecolor{shadecolor}{RGB}{231, 231, 255}
\newenvironment{emp_box}{\begin{shaded}\par\noindent}{\end{shaded}\par\noindent}
\newcounter{problemname}
\newcounter{examplename}
\newenvironment{problem}{\begin{shaded}\stepcounter{problemname}\par\noindent\textbf{题目\arabic{problemname}. }}{\end{shaded}\par}
\newenvironment{example}{\begin{shaded}\stepcounter{examplename}\par\noindent\textbf{例\arabic{examplename}. }}{\end{shaded}\par}
\newenvironment{solution}{\par\noindent\textbf{解答. }}{\par}
\newenvironment{note}{\par\noindent\textbf{题目\arabic{problemname}的注记. }}{\par}

\newtheorem{definition}{定义}[subsection]%跟随三级标题排序
\newtheorem{theorem}{定理}[subsection]%跟随⼆级标题排序改为section
\newtheorem{lemma}[theorem]{Lemma}
\newtheorem{corollary}[theorem]{Corollary}

%\renewcommand{\headrulewidth}{1pt} %页眉线宽,设为0可以去页眉线

\newdimen\doublelineskip % 两横线间的距离
\setlength\doublelineskip{2pt}

\setlength\headheight{21pt}

\fancypagestyle{headings}{
  \renewcommand\headrulewidth{0.4pt}
}
\fancypagestyle{doubleline}[headings]{
  \renewcommand\headrule{%
    \hrule height\headrulewidth width\headwidth%
    \vskip \doublelineskip%
    \hrule height\headrulewidth width\headwidth}
}

%\pagestyle{headings}


\begin{document}

\frontmatter	% 前言部分,页码为小写罗马字母格式;其后的\chapter 不编号。

%\setcounter{secnumdepth}{3}		%增加编号深度
%\setcounter{tocdepth}{3}		%增加目录深度

\tableofcontents

\thispagestyle{empty}
\mainmatter		% 表示开始正文部分的内容 使用数字进行页面编号,对其中的chapter进行编号(第一章、第二章……)

\begin{spacing}{1.0}    % 行间距变为single-space

\setcounter{page}{1}

%\import{D:/mygit/ml/reinforcementLearning/tex/}{front.tex}

%\import{D:/mygit/ml/reinforcementLearning/tex/}{policy.tex}

%\import{D:/mygit/ml/reinforcementLearning/tex/}{variational_inference.tex}

%
%%%%%%%%%%%%%%%%%%%%%%%%%%%%%%%%%%%%%%%%%%%%
% Maximum Entropy
%%%%%%%%%%%%%%%%%%%%%%%%%%%%%%%%%%%%%%%%%%%%

\chapter{Maximum Entropy Reinforcement Learning}

待重读的资源:
\begin{itemize}
%\setlength{\itemsep}{0pt}
%\setlength{\parsep}{0pt}
\setlength{\parskip}{0pt}
\item[-]
\url{https://zhuanlan.zhihu.com/p/269082781}
\end{itemize}

% https://www.zhihu.com/column/c_1268690405400928256

{\bf Entropy is the measure of uncertainty.} The concept originated from 
physics, which it is used as a measure of disorder. It is applicable to 
a statement in physics that an isolated system becomes less organized with 
time, in other words, having a higher entropy.

The principle of maximum entropy states that the probability distribution 
which best represents the current state of knowledge about a system is the 
one with largest entropy, in the context of precisely stated prior data 
(such as a proposition that expresses testable information).

\subsubsection{0. Background: Bellman Equation}

\begin{itemize}
\item
Value function 描述的是:How good is a state?

\item
$Q$-Function 描述的是: How good is a state-action pair?
\end{itemize}

\begin{align*}
V(s) &= \mathbb{E}[G_t | S_t = s] \\
&= \mathbb{E}[R_{t+1} + \gamma R_{t+2} + \gamma^2 R_{t+3} + \ldots | S_t = s] \\
&= \mathbb{E}[R_{t+1} + \gamma (R_{t+2} + \gamma R_{t+3} + \ldots) | S_t = s] \\
&= \mathbb{E}[R_{t+1} + \gamma G_{t+1} | S_t = s] \\
&= \mathbb{E}[R_{t+1} + \gamma V(S_{t+1}) | S_t = s] 
\end{align}
其中$R_{t+1} + \gamma V(S_{t+1})$即所谓 temporal difference target。


% https://blog.csdn.net/qq_27465499/article/details/88817436
\subsubsection{1. Motivation of Maximum Entropy Reinforcement Learning}

Here we explain what {\bf Shannon entropy} is and why adding it to the classic 
reinforcement learning (RL) formulation creates robust agents, ones that 
tolerate unexpected, even adversarial, changes. We focus on intuition for why 
agents with high entropy are robust, in particular to changes in reward or 
changes in dynamics.

Recall that RL aims to solve decision-making problems where a {\bf reward function} 
$r(s,a)$ assigns numerical points to action $a$ in state $s$ of the problem. 
Broadly, the goal is to determine how to score the most points (we'll make 
this statement more concrete later).

The agent's decision-making strategy, also known as its policy, denoted $\pi$, 
can be deterministic or stochastic. When repeatedly confronted with the same 
situation, a deterministic policy will always choose the same option whereas 
a stochastic policy selects according to a probability distribution over options.

For a concrete example of a stochastic policy, think of rock-paper-scissors. 
You wouldn't play a deterministic policy against an intelligent opponent, 
because then they, knowing which option beats yours, could always win. Instead, 
you would make each option equiprobable, minimizing the opponent's ability to 
exploit your strategy.

In truth, a deterministic policy is a stochastic policy; it takes a single 
action with probability $1$. Thus, RL formalism typically takes the policy as 
a distribution over actions given the state, $\pi(a \mid s)$.

Since the policy is randomized, the total reward over the course of an episode 
of $T$ steps is a random variable. To collapse this into a scalar objective, 
we take the expected value. The ideal policy maximizes this objective.
\begin{equation}\label{eq_ideal_policy_rl}
\pi^* = \arg\max_\pi\mathbb{E}_\pi\left[\sum_{t=1}^T r(s_t, a_t)\right].
\end{equation}


对于多峰的$Q$-Value,希望策略能含有概率达到任意一个峰值,而不是只收敛到最高的那个。
而该策略就是最大熵强化学习目标的一个最优解。

数学语言描述(能量形式定义)为:
$\pi(a|s) \propto \exp Q(s,a)$

将熵项加在reward上,设定了最大熵目标,那么求解时就有两条限制:

\begin{itemize}
%\setlength{\itemsep}{0pt}
%\setlength{\parsep}{0pt}
\setlength{\parskip}{0pt}
\item[1)]
$V(s)$, $Q(s,a)$的估计都必须包含熵reward

\item[2)]
策略梯度是根据最大化熵目标推导出来的
\end{itemize}

\subsubsection{2. 强化学习中熵的使用算法分类}

\begin{itemize}
%\setlength{\itemsep}{0pt}
%\setlength{\parsep}{0pt}
\setlength{\parskip}{0pt}
\item[1.]
1991williams,soft Q,soft AC都遵循了上面这两点,是最大熵强化学习。
将熵项加在reward上,其目的是鼓励exploration,希望学到的策略在优化目标的同时尽可能地随机,
同时保持在各个有希望的方向上的可能性,而不是很快greedy收敛到一个局部最优。

\item[2.]
TRPO,PPO方法:都是Policy Gradient方法和神经网络结合下的变体。
使用on-policy更新,所得到的梯度是对真实梯度的一阶近似,因此需要控制步长在合理范围,
步长大了近似就不准确了。于是,在策略梯度后面增加了KL-散度(相对熵)正则项,
目的是控制步长/学习率

\item[3.]
A3C:做法和2类似,宣称的目的和1类似,说自己的想法来自1991williams,
没有更多的理论分析,只有几个实验验证。A3C没有遵循上面提到的两点,
其$V(s)$的估计并没有考虑熵,同时也只是在策略梯度上增加了熵正则项,
并没有理论证明或指出新的梯度优化的目标是什么。这就是最大熵和{\bf 熵正则}的根本区别。
\end{itemize}


\subsubsection{3. SAC 与 soft Q 的区别}

soft $Q$也有策略网络和V,Q网络,但是:

\begin{itemize}
%\setlength{\itemsep}{0pt}
%\setlength{\parsep}{0pt}
\setlength{\parskip}{0pt}
\item[1)]
在更新$Q$-Value时,其target $Q$-Value是一种soft-max $Q$-Value,
而AC框架下target $Q$-Value是下一时刻策略的真实$Q$-Value(一个类似传统$Q$-learning,
另一个类似传统Sarsa)

\item[2)]
在更新$Q$-Value时,soft Q并没有使用真实的relay buffer里面的数据,
而是在其他分布中重新采样,策略网络对$Q$-Value网络没有直接影响
\end{itemize}

{\bf SAC避免了soft Q的复杂性和不稳定性}

\subsubsection{4. SAC的策略更新}

这里有一个关键的限制,就是一旦设定最大熵目标,那么现有的理论结果要求策略只能这样取:
$$
\pi(a|s) \propto \exp Q(s,a)
$$
才能保证最大熵目标, 因此有:
$$
J_\pi(\phi)=E_{s_t\sim D}\left[D_{KL}\left(\pi(\cdot|s_t)\parallel\frac{\exp(Q_\theta(s_t,\cdot))}{Z_\theta(s_t)}\right)\right]
$$



\section{Preliminaries}

In this section, we introduce notation and summarize the standard and 
maximum entropy reinforcement learning frameworks.


\subsection{entropy regularized policy optimization problem}

首先来定义一个熵正则化策略优化问题。在强化学习里面,最重要的一个问题就是
exploration-exploitation 的 tradeoff。下面这个定义就是解决这样的问题的:
$\tau$控制探索和利用的比例,$\tau$越大,熵项成为主导,就更倾向于随机策略(exploration 探索);
$\tau$小的时候奖励项是主导,就更偏向确定性策略(exploitation 利用)。

Given $K$ actions and the corresponding reward vector $\bold q \in E^K$, 
and $\pi$ is a probability distribution in action space. The entropy 
regularized policy optimization problem:
$$
\max_\pi \left\{\pi\bold q + \tau\mathcal{H}(\pi)\right\}
$$
$r\geq 0$ controls the degree of exploration.

Entropy of policy:
$$
\mathcal{H}(\pi) = - \sum_{\pi_a \in \pi} \pi_a\log(\pi_a)
$$
The entropy of deterministic policy is relatively low, and the entropy 
of random policy is relatively high.
\begin{emp_box}
Since $\pi$ is a probability distribution in action space, we have
$$
\sum_{\pi_a \in \pi} \pi_a = 1
$$
\end{emp_box}


\subsection{Softmax function}

下面来定义softmax函数。我们希望不是单纯选择回报最大的动作,
而是对所有的动作都有选择几率(回报大的值多选,反之少选),于是就有这样的softmax函数,
他把一个奖励向量映射成一个不确定性的策略,向量的分量就是选择该动作的概率。

The Softmax $\mathcal{F}_\tau$ function:
$$
\mathcal{F}_\tau (\bold q) = \tau \log \sum_a e^{q_a\tau}
$$

And we define the Soft-Indmax $\bold f_\tau$ function:
$$
\bold f_\tau(\bold q) = \nabla\mathcal{F}_\tau (\bold q) = 
\frac{e^{\bold q/\tau}}{\Sigma_a e^{q_a/\tau}}
= e^{(\bold q-\mathcal{F}_\tau (\bold q))/\tau}
$$

Soft-Indmax function gets the confidence of each action (different from Hardmax
function), and it's possible to explore.


\subsection{the connection}

The connection between entropy regularized policy optimization problem and 
Softmax function:
$$
\mathcal{F}_\tau (\bold q) = \max_\pi \{\pi\bold q + \tau\mathcal{H}(\pi)\}
= \bold f_\tau(\bold q)\cdot \bold q + \tau\mathcal{H}(\bold f_\tau(\bold q))
$$
where $\pi^*=\bold f_\tau(\bold q)$ and
$\mathcal{F}_\tau(\bold q)=q_a - \tau \log \pi_a$, $\forall a$.

The Softmax value is the upper bound on the maximum value, and the gap 
between them is the entropy of the policy.

When $\tau\rightarrow 0$, The entropy regularized policy optimization 
problem becomes the standard expected reward objective, where the optimal 
solution is the hard-max policy.

定义的softmax函数和我们开始定义的熵正则化策略优化问题产生了联系:
softmax函数的值就是该问题的上界,而对应的策略就是soft-indmax给出的策略。
下面是一个简单的证明。

\noindent{\bf Proof:}

The first equation: Let $\mathcal{F}_\tau^*$ denotes as conjugate of 
$\mathcal{F}_\tau$:
$$
\mathcal{F}_\tau^*(\bold p) = \sup_{\bold q}\{\bold p \cdot \bold q - 
\mathcal{F}_\tau(\bold q)\} = \tau\bold p\log\bold p
$$

For $\Sigma_{p\in P}\ p = 1$. Since $\mathcal{F}_\tau$ is closed and 
convex, $\mathcal{F}_\tau = \mathcal{F}_\tau^{**}$:
$$
\mathcal{F}_\tau(\bold q)= \sup_{\bold p}\{\bold p\cdot\bold q - \tau\bold p\log\bold p \}
$$

The second equation uses Lagrange multiplier method:
$$
L = \pi(\bold q - \tau\log\pi) + \lambda(1 - \bold 1\cdot\pi)
$$

KKK condition:
\begin{align*}
1 - \bold 1\cdot\pi &= 0; \\
\tau\log\pi &= \bold q - v(v=\lambda + \tau)
\end{align}




\subsection{Notation}

We address policy learning in continuous action spaces. To that end, we 
consider infinite-horizon Markov decision processes (MDP), defined by 
the tuple $(S, A, p_s, r)$, where the state space $S$ and the action space 
$A$ are assumed to be continuous, and the unknown state transition 
probability $p_s : S \times S \times A \rightarrow [0, \infty)$ represents 
the probability density of the next state $s_{t+1} \in S$ given the current 
state $s_t \in S$ and action $a_t \in A$. The environment emits a reward 
$r : S \times A \rightarrow [r_{\min}, r_{\max}]$ on each transition, which we 
will abbreviate as $r_t \triangleq r(s_t, a_t)$ to simplify notation. We will 
also use $\rho_\pi(s_t)$ and $\rho_\pi(s_t, a_t)$ to denote the state and 
state-action marginals of the trajectory distribution induced by a policy 
$\pi(a_t|s_t)$.

\subsection{Maximum Entropy Reinforcement Learning}

%\usetheme{Madrid}
%\begin{frame}
%\frametitle{Test}
%\setbeamercolor{block title}{fg=white,bg=red!75!black}
%\begin{block}{First Kind Example}
%This is example of the first kind.
%\end{block}
%\setbeamercolor{block title}{fg=white,bg=blue!75!black}
%\begin{block}{Second Kind Example}
%This is example of the second kind.
%\end{block}
%\end{frame}


The standard objective used in reinforcement learning is to maximize the 
expected sum of rewards $\sum_t \mathbb{E}_{(s_t,a_t)\sim\rho_\pi}[r_t]$. 
%
\tcbox[tcbox raise base]{收敛的速度(即求和的项数)也是个问题}%\hfill
%
We will consider a more general maximum entropy objective, which favors
stochastic policies by augmenting the objective with the expected entropy 
of the policy over $\rho_\pi (s_t)$:
%
\begin{equation}\label{entropy_of_policy}
J(\pi) = \sum_{t=0}^{T-1} \mathbb{E}_{(s_t,a_t)\sim\rho_\pi}
[r(s_t, a_t) + \alpha \mathcal{H}(\pi(\cdot | s_t))]
\end{equation}
%
The temperature parameter $\alpha$ determines the relative importance of 
the entropy term against the reward, and thus controls the stochasticity 
of the optimal policy. The maximum entropy objective differs from the 
standard maximum expected reward objective used in conventional reinforcement
learning, though the conventional objective can be recovered in the limit as 
$\alpha \rightarrow 0$. For the rest of this chapter, we will omit writing 
the temperature explicitly, as it can always be subsumed into the reward
by scaling it by $\alpha^{-1}$. The maximum entropy objective has a number 
of conceptual and practical advantages. First, the policy is incentivized 
to explore more widely, while giving up on clearly unpromising avenues. 
Second, the policy can capture multiple modes of near-optimal behavior. In 
particular, in problem settings where multiple actions seem equally 
attractive, the policy will commit equal probability mass to those actions. 
Lastly, prior work has observed substantially improved exploration from 
this objective (Haarnoja et al., 2017; Schulman et al., 2017), and in our 
experiments, we observe that it considerably improves learning speed over 
state-of-art methods that optimize the conventional objective function. If 
we wish to extend the objective to infinite horizon problems, it is convenient 
to also introduce a discount factor $\gamma$ to ensure that the sum of 
expected rewards and entropies is finite. Writing down the precise maximum 
entropy objective for the infinite horizon discounted case is more involved 
(Thomas, 2014) and is deferred to Appendix A.

If we admit stochastic policies, reinforcement learning becomes a great 
application area for information theoretic concepts.

To that end, let's think about how the entropy of an agent's policy influences 
its actions. Entropy creates surprisal, so an agent with high entropy will 
actually have multiple strategies at its disposal simply by nature of policy's 
randomness. In addition, we could interpret a maximum entropy policy 
distribution as making fewer assumptions about the world it's in.

Thus, we hypothesize that encouraging entropy in our agents has beneficial 
properties. This formulation is exactly MaxEnt RL. We'll include an entropy 
term as a bonus in the optimization. Added such entropy term to the equation
\ref{eq_ideal_policy_rl} we have:
\begin{equation}\label{eq_ideal_maxent_rl}
\pi^* = \arg\max_\pi\left(\mathbb{E}_\pi\left[\sum_{t=1}^T r(s_t, a_t)\right]
\textcolor{red}{+ \mathcal{H}(\pi(a|s))}\right).
\end{equation}


\subsection{Why is the softmax function called that way?}

% https://math.stackexchange.com/questions/1888141/why-is-the-softmax-function-called-that-way

\begin{emp_box}
\noindent{I} understand that the function "squashes" a real vector space between the 
values 0 and 1.

However I don't see what this has to do with the "max" function, or why that 
makes it a "softer" version of the max function.
\end{emp_box}

The largest element in the input vector remains the largest element after 
the softmax function is applied to the vector, hence the "max" part. The 
"soft" signifies that the function keeps information about the other, 
non-maximal elements in a reversible way (as opposed to a "hardmax", which 
is just the standard maximum function).

The function produces a probability distribution from any vector, and is 
thus used in machine learning when inputs need to be classified. The output 
of a neural network is normalised via this function, and this normalisation 
is required for machine learning techniques to work.
The softmax function is defined on a vector and outputs a vector.

The "hardmax" function takes a vector and sets its largest element to 1, and 
all others to 0. The softmax function does almost the same thing, but it is 
continuous, and most machine learning techniques require this property to 
train neural networks, hence the "soft" modifier.

I always thought it was called softmax because it is differentiable ("soft") 
at all points for all elements of the input vector. This explanation would 
be analogous to what makes the softplus function,
$f(x) = \ln(1 + e^x)$, the "soft" version of
$f(x) = \max(0, x)$



\section{连续的情况}

% https://blog.csdn.net/weixin_44044411/article/details/119922423

最大熵学习是一种Feature Matching的方法,其采用了最大熵原则消除匹配歧义。
该原则基于一种假设——即专家系统轨迹生成自己的专家特征期望的策略是最优轨迹。

正态分布是所有概率分布中熵最大的(原始会议presentation里面说均匀分布的信息熵最大,
确认过是个{\bf 错误结论},试想一下均匀分布的分布区间有限而正态无限)。

\begin{emp_box}
这里个人认为离散情况下均匀分布的信息熵最大,连续变量时则正态分布的信息熵最大。
\end{emp_box}

熵越大,先验信息越少。


\subsection{算法原理}

连续变量而言,信息熵通常表示为:
$$
Ent = \int_{x\sim\pi}-p(x)\log p(x)
$$

对于强化学习任务而言,最大化信息熵写为:
$$
\max\sum_{\zeta\in D}-P(\zeta|\theta)\log P(\zeta | \theta))
$$

$$
s.t. 
\begin{cases}
\Sigma_{\zeta\in D} P(\zeta|\theta)f_\zeta = \hat{f} \\
\Sigma_{\zeta\in D} P(\zeta|\theta) = 1
\end{cases}
$$

构造拉格朗日函数
$$
L(P, \lambda, \mu) = \sum_{\zeta\in D} \left[P(\zeta|\theta)\log P(\zeta | \theta) 
+ \lambda (P(\zeta|\theta)f_\zeta - \hat{f})
+ \mu(P(\zeta|\theta) - 1)\right]
$$

应用拉格朗日函数的KKT条件:
\begin{subequations}\label{max_entropy_lagrange}
\begin{align}
\nabla L_P &= \Sigma_{\zeta\in D}\log P(\zeta | \theta) + 1 + \lambda f_\zeta + \mu = 0 
\tag{\ref{max_entropy_lagrange}{a}}\label{max_entropy_lagrange_a} \\
\nabla L_\lambda &= \Sigma_{\zeta\in D} P(\zeta | \theta)f_\zeta - \hat{f} = 0 
\tag{\ref{max_entropy_lagrange}{b}}\label{max_entropy_lagrange_b} \\
\nabla L_\mu &= \Sigma_{\zeta\in D} P(\zeta | \theta) - 1 = 0
\tag{\ref{max_entropy_lagrange}{c}}\label{max_entropy_lagrange_c}
\end{align}
\end{subequations}

由公式(\ref{max_entropy_lagrange_a})及(\ref{max_entropy_lagrange_c})得
$$
P(\zeta | \theta) = 
\frac{\exp(-1 -\mu - \lambda f_\zeta)}{\Sigma_{\zeta\in D}\exp(-1 - \mu - \lambda f_\zeta)}
$$

光靠这个式子肯定是解不出最优的$\theta$的,这就需要另一个假设——
{\bf 使用$\theta$参数的Reward函数$R_\theta(\tau)$时,
$\zeta$轨迹的概率$P(\zeta|\theta)$正比于$R_{\theta}(\zeta)$的自然指数},
再加上概率归一性约束,可得专家系统策略的轨迹概率为:
$$
P(\zeta | \theta) = 
\frac{\exp(R_\theta(\zeta))}{\int_{\tau\in D}\exp(R_\theta(\tau))d\tau}
$$
需要注意的是,这里的$R(\theta)$指的是累积奖赏而非单步奖赏。






\import{D:/mygit/ml/reinforcementLearning/tex/}{entropy.tex}

\import{D:/mygit/ml/reinforcementLearning/tex/}{sac.tex}

% 逆强化学习
%\import{D:/mygit/ml/reinforcementLearning/tex/}{inverse_rl.tex}

\chapter{附录}
% 附件内容:最大熵
\import{D:/mygit/ml/reinforcementLearning/tex/}{entropy_max.tex}

%\import{D:/mygit/ml/reinforcementLearning/tex/}{back.tex}

\printbibliography

\end{document}
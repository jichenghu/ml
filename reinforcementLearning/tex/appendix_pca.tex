

%+++++++++++++++++++++++++++++++++++++++++++
\section{PCA}
%-------------------------------------------
% file:///F:/mygit/ml/reinforcementLearning/refs/dgm/TFG_final.pdf

Principal Component Analysis (known as PCA) \cite{Shlens2014} is a technique used for 
dimensionality reduction, lossy data compression and data visualization that has 
wide-ranging applications in fields such as signal processing, mechanical engineering or 
data analysis.
Depending on the field scope, it can also be named discrete Karthunen-Loéve transform.

PCA is a technique that can be analyzed from two different points of view: linear
algebra and probability. In this section it is presented an overview of PCA from the
probabilistic point of view as it is closely related to the simplest case of the variational
autoencoder and a good starting point to latent variable models. The explanation is largely
based on Bishop's book \cite{Bishop2006} page 570-587.

Although PCA was firstly conceived by Karl Pearson in 1901, it was not until
the late 90s when probabilistic PCA was formulated by Tipping and Bishop and
by Roweis. Probabilistic PCA holds several advantages compared to conventional
PCA. One of them is that it can be used as a simple generative model and this is why this
algorithm is meaningful in this tech report.

As its name states, PCA finds the principal components of data, or in other words,
it finds the features, the directions where there is the most variance. So given a dataset 
$X = \{x^{(i)}\}^N_1$ with dimensionality $D_x$, the main idea behind PCA is to obtain a subspace
(called the principal-component subspace) with dimension $D_z$, being $D_z << D_x$, so that
each $x^{(i)}$ is represented with a $z^{(i)}$ (latent variable) in the best possible way. Then, it is
possible to recover $x^{(i)}$ from $z^{(i)}$.
$$
x^{(i)} \in \mathcal{R}^{D_x} \, \rightarrow \, z^{(i)} \in \mathcal{R}^{D_z}
$$

The probabilistic PCA is a maximum likelihood solution of a probabilistic latent variable model in 
which all marginal and conditional distributions are Gaussian. The likelihood function to be 
maximized is expressed as:
\begin{align}
p(x) = \int p(x | z) p(z) dz
\end{align}
As all distributions are Gaussian, the probability distribution of $x$ can be expressed
as $p(x) = N(x | \mu, C)$ where the covariance matrix is $C = WW^T + \sigma^2$. Therefore, 
the parameters of the model are the $D_x x D_z$ matrix $W$, the $D_x$ dimensional vector 
$\mu$ and the scalar $\sigma^2$.




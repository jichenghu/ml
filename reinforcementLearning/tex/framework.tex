\chapter{RL Framework}

强化学习框架代码

关键类:
\begin{itemize}
\item[-]
环境 `env`

\item[-]
模型 `model`

\item[-]
算法 `algorithm`

\item[-]
agent `agent`

\item[-]
回放内存 `rpm`
\end{itemize}


关键技术:
\begin{itemize}
\item
方法的多态:如同调用父类中定义的 virtual 方法,子类中有不同的实现。
	\begin{itemize}
	\item
	python 中不具备类似的类的 virtual 方法

	\item
	可将不同的对象放入同一个列表,遍历列表,调用对象的相同方法名的函数。
	这种方式甚至可以调用不同类但具有相同方法名的对象(这些不同类可以无相同父类)

	\item
	可以把父类的子类的对象封装进一个链表,对父类 virtual 方法的调用对应为对链表中对象的同名方法的调用。
	\end{itemize}

\item
Assign functions to a variable:
\begin{lstlisting}[language=Python]
# function defined
def multiply_num(a):
	b = 40
	r = a*b
	return r


# drivercode
# assigning function
z = multiply_num

# invoke function
print(z(6))
print(z(10))
print(z(100))
\end{lstlisting}

\end{itemize}


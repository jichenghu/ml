

\section{Clarify}

\subsection{on-policy, off-policy, policy-based, value-based, model-based, model-free}

% https://stats.stackexchange.com/questions/407230/what-is-the-difference-between-policy-based-on-policy-value-based-off-policy

Here is a quick summary on the Reinforcement Learning taxonomy:

\subsubsection{On-policy vs. Off-Policy}

This division is based on whether you update your $Q$-values based on actions undertaken 
according to your current policy or not. Let's say your current policy is a completely 
random policy. You're in state $s$ and make an action $a$ that leads you to state $s′$. 
Will you update your $Q(s,a)$ based on the best possible action you can take in $s′$ or 
based on an action according to your current policy (random action)? The first choice 
method is called off-policy and the latter - on-policy. E.g. $Q$-learning does the first 
and SARSA does the latter.

\subsubsection{Policy-based vs. Value-based}

In Policy-based methods we explicitly build a representation of a policy (mapping $\pi: 
s \rightarrow a$) and keep it in memory during learning.

In Value-based we don't store any explicit policy, only a value function. The policy is 
here implicit and can be derived directly from the value function (pick the action with 
the best value).

{\bf Actor-critic is a mix of the two.}

\subsubsection{Model-based vs. Model-free}

The problem we're often dealing with in RL is that whenever you are in state $s$ and make 
an action $a$ you might not necessarily know the next state $s′$ that you'll end up in 
(the environment influences the agent).

In {\bf Model-based} approach you either have an access to the model (environment) so you 
know the probability distribution over states that you end up in, or you first try to build 
a model (often - approximation) yourself. This might be useful because it allows you to do 
planning (you can "think" about making moves ahead without actually performing any actions).

In {\bf Model-free} you're not given a model and you're not trying to explicitly figure out 
how it works. You just collect some experience and then derive (hopefully) optimal policy.




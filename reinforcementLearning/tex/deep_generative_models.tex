
% https://deepgenerativemodels.github.io/

\chapter{Deep Generative Models}
\setcounter{section}{-1}

%+++++++++++++++++++++++++++++++++++++++++++
\section{Overview}
%-------------------------------------------
% https://deepgenerativemodels.github.io/

Generative models are widely used in many subfields of AI and Machine Learning. Recent 
advances in parameterizing these models using deep neural networks, combined with 
progress in stochastic optimization methods, have enabled scalable modeling of complex, 
high-dimensional data including images, text, and speech. In this chapter, we will 
study the probabilistic foundations and learning algorithms for deep generative models, 
including variational autoencoders, generative adversarial networks, autoregressive 
models, normalizing flow models, energy-based models, and score-based models. This 
chapter will also discuss application areas that have benefitted from deep generative 
models, including computer vision, speech and natural language processing, graph mining, 
reinforcement learning, reliable machine learning, and inverse problem solving.

Intelligent agents are constantly generating, acquiring, and processing
data. This data could be in the form of {\em images} that we capture on our
phones, {\em text} messages we share with our friends, {\em graphs} that model
interactions on social media, {\em videos} that record important events,
etc. Natural agents excel at discovering patterns, extracting
knowledge, and performing complex reasoning based on the data they observe. How
can we build artificial learning systems to do the same?

In this chapter, we will study generative models that view the world under the lens of probability.
In such a worldview, we can think of any kind of
observed data, say $\mathcal{D}$, as a finite set of samples from an
underlying distribution, say $p_{\mathrm{data}}$. At its very core, the
goal of any generative model is then to approximate this data
distribution given access to the dataset $\mathcal{D}$. The hope is that
if we are able to {\em learn} a good generative model, we can use the
learned model for downstream {\em inference}.


%+++++++++++++++++++++++++++++++++++++++++++
\subsection{Learning}

We will be primarily interested in parametric approximations to the data
distribution, which summarize all the information about the dataset $\mathcal{D}$ in
a finite set of parameters. In contrast with non-parametric models,
parametric models scale more efficiently with large datasets but are
limited in the family of distributions they can represent.

In the parametric setting, we can think of the task of learning a
generative model as picking the parameters within a family of model
distributions that minimizes some notion of distance\footnote{
  As we shall see later, functions that do not satisfy all 
  properties of a distance metric are also used in practice, e.g., KL
  divergence.} 
between the model distribution and the data distribution.

%\begin{comment}
\begin{figure}[H]
	\centering
	\begin{minipage}{0.49\linewidth}
		\centering
		\includegraphics[scale=0.518]{pix/dgm/learning_1.png}
	\end{minipage}
	%\qquad
	\begin{minipage}{0.49\linewidth}
		\centering
		\includegraphics[scale=0.518]{pix/dgm/learning_2.png}
	\end{minipage}
\end{figure}
%\end{comment}

For instance, we might be given access to a dataset of dog images $\mathcal{D}$ and
our goal is to learn the parameters of  a generative model $\theta$ within a model 
family $\mathcal{M}$ such that
the model distribution $p_\theta$ is close to the data distribution over dogs
$p_{\mathrm{data}}$. Mathematically, we can specify our goal as the
following optimization problem: 
\begin{equation}
\min_{\theta\in \mathcal{M}}d(p_{\mathrm{data}}, p_{\theta})
\label{eq:learning_gm}
%\tag{1}
\end{equation}
where $p_{\mathrm{data}}$ is accessed via the dataset
$\mathcal{D}$ and $d(\cdot)$ is a notion of distance between probability distributions.

As we navigate through this chapter, it is interesting to take note of
the difficulty of the problem at hand. A typical image from a modern
phone camera has a resolution of approximately $700 \times 1400$ pixels.
Each pixel has three channels: R(ed), G(reen) and B(lue) and each
channel can take a value between $0$ to $255$. Hence, the number of possible
images is given by $256^{700 \times 1400 \times 3}\approx 10 ^{800000}$.
In contrast, ImageNet, one of the largest publicly available datasets,
consists of only about 15 million images. Hence, learning a generative
model with such a limited dataset is a highly underdetermined problem.

Fortunately, the real world is highly structured and automatically
discovering the underlying structure is key to learning generative
models. For example, we can hope to learn some basic artifacts about
dogs even with just a few images: two eyes, two ears, fur etc. Instead
of incorporating this prior knowledge explicitly, we will hope the model
learns the underlying structure directly from data. There is no free
lunch however, and indeed successful learning of generative models will
involve instantiating the optimization problem in
$(\ref{eq:learning_gm})$ in a suitable way. In this chapter, we will be
primarily interested in the following questions:

\begin{itemize}
%\setlength{\itemsep}{0pt}
%\setlength{\parsep}{0pt}
\setlength{\parskip}{0pt}
\item[-]
What is the representation for the model family $\mathcal{M}$?

\item[-]
What is the objective function $d(\cdot)$?

\item[-]
What is the optimization procedure for minimizing $d(\cdot)$?
\end{itemize}

In the next few sections, we will take a deeper dive into certain
families of generative models. For each model family, we will note how
the representation is closely tied with the choice of learning objective
and the optimization procedure.


%+++++++++++++++++++++++++++++++++++++++++++
\subsection{Inference}

For a discriminative model such as logistic regression, the fundamental
inference task is to predict a label for any given datapoint. Generative
models, on the other hand, learn a joint distribution over the entire
data.\footnote{Technically, a probabilistic discriminative model is also a
    generative model of the labels conditioned on the data. However, the
    usage of the term generative models is typically reserved for high
    dimensional data.}

While the range of applications to which generative models have been
used continue to grow, we can identify three fundamental inference
queries for evaluating a generative model.:

\begin{itemize}
%\setlength{\itemsep}{0pt}
%\setlength{\parsep}{0pt}
\setlength{\parskip}{0pt}
\item[1.]
{\bf Density estimation:} Given a datapoint $\mathbf{x}$, what is the
    probability assigned by the model, i.e., $p_\theta(\mathbf{x})$?

\item[2.]
{\bf Sampling:} How can we {\bf generate} novel data from the model
    distribution, i.e.,
    $\mathbf{x}_{\mathrm{new}} \sim p_\theta(\mathbf{x})$?

\item[3.]
{\bf Unsupervised representation learning:} How can we learn meaningful
    feature representations for a datapoint $\mathbf{x}$?
\end{itemize}

Going back to our example of learning a generative model over dog
images, we can intuitively expect a good generative model to work as
follows. For density estimation, we expect $p_\theta(\mathbf{x})$ to be
high for dog images and low otherwise. Alluding to the name {\bf generative
model}, sampling involves generating novel images of dogs beyond the
ones we observe in our dataset. Finally, representation learning can
help discover high-level structure in the data such as the breed of
dogs.

In light of the above inference tasks, we note two caveats. First,
quantitative evaluation of generative models on these tasks is itself
non-trivial (in particular, sampling and representation learning) and an
area of active research. Some quantitative metrics exist, but these
metrics often fail to reflect desirable qualitative attributes in the
generated samples and the learned representations. Secondly, not all
model families permit efficient and accurate inference on all these
tasks. Indeed, the trade-offs in the inference capabilities of the
current generative models have led to the development of very diverse approaches as
we shall see in this course.


%+++++++++++++++++++++++++++++++++++++++++++
\section{Autoregressive Models}
%-------------------------------------------

We begin our study into generative modeling with autoregressive models. As before, 
we assume we are given access to a dataset $\mathcal{D}$ of $n$-dimensional 
datapoints $\mathbf{x}$. For simplicity, we assume the datapoints are binary, i.e., 
$\mathbf{x} \in \{0,1\}^n$.


%+++++++++++++++++++++++++++++++++++++++++++
\subsection{Representation}

By the chain rule of probability, we can factorize the joint distribution over the 
$n$-dimensions as 
\begin{equation}
\label{eq:chain_rule}
p(\mathbf{x}) = \prod\limits_{i=1}^{n}p(x_i \vert x_1, x_2, \ldots, x_{i-1}) = 
\prod\limits_{i=1}^{n} p(x_i \vert \mathbf{x}_{< i } )
\end{equation}
where $\mathbf{x}_{< i}=[x_1, x_2, \ldots, x_{i-1}]$ denotes the vector of random 
variables with index less than $i$. 

The chain rule factorization can be expressed graphically as a Bayesian network.

\begin{figure}[H]
\centering
\includegraphics[scale=0.8]{pix/dgm/autoregressive.png}
\caption{Graphical model for an autoregressive Bayesian network with no 
conditional independence assumptions.}
%\label{fig:label}
\end{figure}

Such a Bayesian network that makes no conditional independence assumptions is said 
to obey the *autoregressive* property.

The term {\bf autoregressive} originates from the literature on time-series models 
where observations from the previous time-steps are used to predict the value at 
the current time step. Here, we fix an ordering of the variables $x_1, x_2, \ldots, 
x_n$ and the distribution for the $i$-th random variable depends on the values of 
all the preceding random variables in the chosen ordering $x_1, x_2, \ldots, x_{i-1}$.

If we allow for every conditional $p(x_i \vert \mathbf{x}_{< i})$ to be specified 
in a tabular form, then such a representation is fully general and can represent 
any possible distribution over $n$ random variables. However, the space complexity 
for such a representation grows exponentially with $n$.

To see why, let us consider the conditional for the last dimension, given by 
$p(x_n \vert \mathbf{x}_{< n})$. In order to fully specify this conditional, we need 
to specify a probability distribution for each of the $2^{n-1}$ configurations of 
the variables $x_1, x_2, \ldots, x_{n-1}$. For any one of the $2^{n-1}$ possible 
configurations of the variables, the probabilities should sum to one. Therefore, we 
need only one parameter for each configuration, so the total number of parameters 
for specifying this conditional is given by $2^{n-1}$. Hence, a tabular 
representation for the conditionals is impractical for learning the joint 
distribution factorized via chain rule in (\ref{eq:chain_rule}).

In an {\bf autoregressive generative model}, the conditionals are specified as 
parameterized functions with a fixed number of parameters. That is, we assume the 
conditional distributions $p(x_i \vert \mathbf{x}_{<i})$ to correspond to a Bernoulli 
random variable and learn a function that maps the preceding random variables 
$x_1, x_2, \ldots, x_{i-1}$ to the mean of this distribution. Hence, we have
$$
p_{\theta_i}(x_i \vert \mathbf{x}_{< i}) = \mathrm{Bern}(f_i(x_1, x_2, \ldots, x_{i-1}))
$$
where $\theta_i$ denotes the set of parameters used to specify the mean function 
$f_i: \{0,1\}^{i-1}\rightarrow [0,1]$. The term \textit{autoregressive} originates from 
the literature on time-series models where observations from the previous time-steps are 
used to predict the value at the current time step. Here, we are predicting the 
distribution for the $i$-th random variable using the values of the preceeding random 
variables in the sequence $x_1, x_2, \ldots, x_n$.

The number of parameters of an autoregressive generative model are given by 
$\sum_{i=1}^n \vert \theta_i \vert$. As we shall see in the examples below, the number 
of parameters are much fewer than the tabular setting considered previously. Unlike the 
tabular setting however, an autoregressive generative model cannot represent all 
possible distributions. Its expressiveness is limited by the fact that we are limiting 
the conditional distributions to correspond to a Bernoulli random variable with the 
mean specified via a restricted class of parameterized functions.

\begin{figure}[H]
\centering
\includegraphics[scale=0.618]{pix/dgm/fvsbn.png}
\caption{A fully visible sigmoid belief network over four variables. \\ The conditionals 
are denoted by \(\widehat{x}_1, \widehat{x}_2, \widehat{x}_3, \widehat{x}_4\) 
respectively.}
%\label{fig:label}
\end{figure}
In the simplest case, we can specify the function as a linear combination of the input 
elements followed by a sigmoid non-linearity (to restrict the output to lie between 0 
and 1). This gives us the formulation of a \textit{fully-visible sigmoid belief network} 
(\href{https://papers.nips.cc/paper/1153-does-the-wake-sleep-algorithm-produce-good-density-estimators.pdf}{FVSBN}).
$$
f_i(x_1, x_2, \ldots, x_{i-1}) =\sigma\left(\alpha^{(i)}_0 + \alpha^{(i)}_1 x_1 + \ldots + 
\alpha^{(i)}_{i-1} x_{i-1}\right)  
$$
where $\sigma$ denotes the sigmoid function and $\theta_i=\{\alpha^{(i)}_0,\alpha^{(i)}_1, 
\ldots, \alpha^{(i)}_{i-1}\}$ denote the parameters of the mean function. The conditional 
for variable $i$ requires $i$ parameters, and hence the total number of parameters in 
the model is given by $\sum_{i=1}^ni= O(n^2)$. Note that the number of parameters are 
much fewer than the exponential complexity of the tabular case.

A natural way to increase the expressiveness of an autoregressive generative model is to 
use more flexible parameterizations for the mean function e.g., multi-layer perceptrons 
(MLP). For example, consider the case of a neural network with 1 hidden layer. The mean 
function for variable $i$ can be expressed as
$$
\mathbf{h}_i = \sigma(A_i \mathbf{x_{< i}} + \mathbf{c}_i)
f_i(x_1, x_2, \ldots, x_{i-1}) =\sigma(\boldsymbol{\alpha}^{(i)}\mathbf{h}_i +b_i )
$$
where $\mathbf{h}_i \in \mathbb{R}^d$ denotes the hidden layer activations for the MLP 
and $\theta_i = \{A_i \in \mathbb{R}^{d\times (i-1)},  \mathbf{c}_i \in \mathbb{R}^d, 
\boldsymbol{\alpha}^{(i)}\in \mathbb{R}^d, b_i \in \mathbb{R}\}$ are the set of parameters 
for the mean function $\mu_i(\cdot)$. The total number of parameters in this model is 
dominated by the matrices $A_i$ and given by $O(n^2 d)$.

\begin{figure}[H]
\centering
\includegraphics[scale=0.618]{pix/dgm/nade.png}
\caption{A neural autoregressive density estimator over four variables. The conditionals 
are denoted by \(\widehat{x}_1, \widehat{x}_2, \widehat{x}_3, \widehat{x}_4\) 
respectively. The blue connections denote the tied weights \(W[., i]\) used for 
computing the hidden layer activations.}
%\label{fig:label}
\end{figure}

The {\em Neural Autoregressive Density Estimator} (
\href{http://proceedings.mlr.press/v15/larochelle11a/larochelle11a.pdf}{NADE}) provides 
an alternate MLP-based parameterization that is more statistically and computationally 
efficient than the vanilla approach. In NADE, parameters are shared across the functions 
used for evaluating the conditionals. In particular, the hidden layer activations are 
specified as
$$
\mathbf{h}_i = \sigma(W_{., <i} \mathbf{x_{<i}} + \mathbf{c})\\
f_i(x_1, x_2, \ldots, x_{i-1}) =\sigma(\boldsymbol{\alpha}^{(i)}\mathbf{h}_i +b_i )  
$$
where $\theta=\{W\in \mathbb{R}^{d\times n}, \mathbf{c} \in \mathbb{R}^d, \{\boldsymbol{
\alpha}^{(i)}\in \mathbb{R}^d\}^n_{i=1}, \{b_i \in \mathbb{R}\}^n_{i=1}\}$ is the full set 
of parameters for the mean functions $f_1(\cdot), f_2(\cdot), \ldots, f_n(\cdot)$. The 
weight matrix $W$ and the bias vector $\mathbf{c}$ are shared across the conditionals. 
Sharing parameters offers two benefits:
\begin{enumerate}
\item[1.] 
The total number of parameters from $O(n^2 d)$ to $O(nd)$ [readers are encouraged 
to check!].

\item[2.] 
The hidden unit activations can be evaluated in $O(nd)$ time via the following 
recursive strategy:
$$
\mathbf{h}_i = \sigma(\mathbf{a}_i)\\
\mathbf{a}_{i+1} = \mathbf{a}_{i} + W[\cdot, i]x_i
$$
with the base case given by $\mathbf{a}_1=\mathbf{c}$.
\end{enumerate}


\subsubsection{Extensions to NADE}

The \href{https://arxiv.org/abs/1306.0186}{RNADE} algorithm extends NADE to learn 
generative models over real-valued data. Here, the conditionals are modeled via a 
continuous distribution such as a equi-weighted mixture of $K$ Gaussians. Instead 
of learning a mean function, we now learn the means $\mu_{i,1}, \mu_{i,2},\ldots, 
\mu_{i,K}$ and variances $\Sigma_{i,1}, \Sigma_{i,2},\ldots, \Sigma_{i,K}$ of the 
$K$ Gaussians for every conditional. For statistical and computational efficiency, 
a single function $g_i: \mathbb{R}^{i-1}\rightarrow\mathbb{R}^{2K}$ outputs all 
the means and variances of the $K$ Gaussians for the $i$-th conditional distribution.

Notice that NADE requires specifying a single, fixed ordering of the variables. The 
choice of ordering can lead to different models. 
The \href{https://arxiv.org/abs/1310.1757}{EoNADE} algorithm allows training an 
ensemble of NADE models with different orderings.


%+++++++++++++++++++++++++++++++++++++++++++
\subsection{Learning and inference}

Recall that learning a generative model involves optimizing the closeness between the 
data and model distributions. One commonly used notion of closeness in the KL 
divergence between the data and the model distributions.
\begin{align*}
\min_{\theta\in \mathcal{M}}d_{KL}
(p_{\mathrm{data}}, p_{\theta}) = \mathbb{E}_{\mathbf{x} \sim p_{\mathrm{data}} }
\left[\log p_{\mathrm{data}}(\mathbf{x}) - \log p_{\theta}(\mathbf{x})\right]
\end{align}

Before moving any further, we make two comments about the KL divergence. First, we note 
that the KL divergence between any two distributions is asymmetric. As we navigate through 
this chapter, the reader is encouraged to think what could go wrong if we decided to 
optimize the reverse KL divergence instead\footnote{这里是用 $p_\theta$ 的采样去逼近 
$p_{\mathrm{data}}$的分布,the reverse KL divergence 则是用 $p_{\mathrm{data}}$ 的采样去逼近 
$p_\theta$ 的分布}. Secondly, the KL divergences heavily penalizes 
any model distribution $p_\theta$ which assigns low probability to a datapoint that is 
likely to be sampled under $p_{\mathrm{data}}$. In the extreme case, if the density 
$p_\theta(\mathbf{x})$ evaluates to zero for a datapoint sampled from $p_{\mathrm{data}}$, 
the objective evaluates to $+\infty$.

Since $p_{\mathrm{data}}$ does not depend on $\theta$, we can equivalently recover the 
optimal parameters via maximizing likelihood estimation.
\begin{align*}
\max_{\theta\in \mathcal{M}}\mathbb{E}_{\mathbf{x} \sim p_{\mathrm{data}} }
\left[\log p_{\theta}(\mathbf{x})\right].
\end{align*}
Here, $\log p_{\theta}(\mathbf{x})$ is referred to as the log-likelihood of the datapoint 
$\mathbf{x}$ with respect to the model distribution $p_\theta$.

To approximate the expectation over the unknown $p_{\mathrm{data}}$, we make an assumption: 
points in the dataset $\mathcal{D}$ are sampled i.i.d\footnote{独立同分布,
independent and identically distributed}. from $p_{\mathrm{data}}$. This 
allows us to obtain an unbiased Monte Carlo estimate of the objective as
\begin{align}
\label{eq:mle}
\max_{\theta\in \mathcal{M}}\frac{1}{\vert D \vert} 
\sum_{\mathbf{x} \in\mathcal{D} }\log p_{\theta}(\mathbf{x}) 
= \mathcal{L}(\theta \vert \mathcal{D}).
\end{align}

The maximum likelihood estimation (MLE) objective has an intuitive interpretation: pick 
the model parameters $\theta \in \mathcal{M}$ that maximize the log-probability of the 
observed datapoints in $\mathcal{D}$.

In practice, we optimize the MLE objective using mini-batch gradient ascent. The algorithm 
operates in iterations. At every iteration $t$, we sample a mini-batch $\mathcal{B}_t$ 
of datapoints sampled randomly from the dataset ($\vert \mathcal{B}_t\vert < \vert 
\mathcal{D} \vert$) and compute gradients of the objective evaluated for the mini-batch. 
These parameters at iteration $t+1$ are then given via the following update rule
$$
\theta^{(t+1)} = \theta^{(t)} + r_t \nabla_\theta\mathcal{L}(\theta^{(t)} \vert 
\mathcal{B}_t)
$$
where $\theta^{(t+1)}$ and $\theta^{(t)}$ are the parameters at iterations $t+1$ and $t$ 
respectively, and $r_t$ is the learning rate at iteration $t$. Typically, we only specify 
the initial learning rate $r_1$ and update the rate based on a schedule. 
\href{http://cs231n.github.io/optimization-1/}{Variants} of stochastic gradient ascent, 
such as RMS prop and Adam, employ modified update rules that work slightly better in 
practice.

From a practical standpoint, we must think about how to choose hyperparameters (such as 
the initial learning rate) and a stopping criteria for the gradient descent. For both 
these questions, we follow the standard practice in machine learning of monitoring the 
objective on a validation dataset. Consequently, we choose the hyperparameters with the 
best performance on the validation dataset and stop updating the parameters when the 
validation log-likelihoods cease to improve\footnote{Given the non-convex nature of such 
problems, the optimization procedure can get stuck in local optima. Hence, early 
stopping will generally not be optimal but is a very practical strategy.}.

Now that we have a well-defined objective and optimization procedure, the only remaining 
task is to evaluate the objective in the context of an autoregressive generative model. 
To this end, we substitute the factorized joint distribution in 
Eq.$~\ref{eq:chain_rule}$ of an autoregressive model 
in the MLE objective in Eq.$~\ref{eq:mle}$ to get
$$
\max_{\theta \in \mathcal{M}}\frac{1}{\vert D \vert} \sum_{\mathbf{x} \in\mathcal{D} }
\sum_{i=1}^n\log p_{\theta_i}(x_i \vert \mathbf{x}_{< i})
$$
where $\theta = \{\theta_1, \theta_2, \ldots, \theta_n\}$ now denotes the
collective set of parameters for the conditionals.

Inference in an autoregressive model is straightforward. For density estimation of an 
arbitrary point $\mathbf{x}$, we simply evaluate the log-conditionals 
$\log p_{\theta_i}(x_i \vert \mathbf{x}_{< i})$ for each $i$ and add these up to obtain 
the log-likelihood assigned by the model to $\mathbf{x}$. Since we know conditioning 
vector $\mathbf{x}$, each of the conditionals can be evaluated in parallel. Hence, density 
estimation is efficient on modern hardware.

Sampling from an autoregressive model is a sequential procedure. Here, we first sample 
$x_1$, then we sample $x_2$ conditioned on the sampled $x_1$, followed by $x_3$ 
conditioned on both $x_1$ and $x_2$ and so on until we sample $x_n$ conditioned on the 
previously sampled $\mathbf{x}_{< n}$. For applications requiring real-time generation 
of high-dimensional data such as audio synthesis, the sequential sampling can be an 
expensive process. Later in this chapter, we will discuss how parallel WaveNet, an 
autoregressive model sidesteps this expensive sampling process.

% TODO: add NADE samples figure

Finally, an autoregressive model does not directly learn unsupervised representations of 
the data. In the next few set of lectures, we will look at latent variable models (e.g., 
variational autoencoders) which explicitly learn latent representations of the data.

% TODO: Autoregressive generative models based on Autoencoders, RNNs, and CNNs.
% MADE, Char-RNN, Pixel-CNN, Wavenet

% Additional parameterizations
% ==============
% Coming soon: MADE, Char-RNN, Pixel-CNN, Wavenet -->




\newcommand{\D}{\mathcal{D}}
\newcommand{\KL}[2]{D_\mathrm{KL}\paren{#1 \mathbin{\|} #2}}
\renewcommand{\P}{\mathcal{P}}
\newcommand{\X}{\mathcal{X}}
\newcommand{\Z}{\mathcal{Z}}
\newcommand{\Q}{\mathcal{Q}}
\newcommand{\bx}{\mathbf{x}}
\newcommand{\M}{\mathcal{B}}
\newcommand{\ELBO}{\mathrm{ELBO}}
\newcommand{\bz}{\mathbf{z}}
\newcommand{\giv}{\mid}
\newcommand{\paren}[1]{\left(#1\right)}
\newcommand{\brac}[1]{\left[#1\right]}
\newcommand{\veps}{\varepsilon}
\newcommand{\set}[1]{\left\{#1\right\}}
\renewcommand{\d}{\mathop{}\!\mathrm{d}}
\newcommand{\Expect}{\mathbb{E}}
\newcommand{\Normal}{\mathcal{N}}
\newcommand{\I}{\mathbf{I}}
\newcommand{\0}{\mathbf{0}}

%+++++++++++++++++++++++++++++++++++++++++++
\section{Variational Auto-Encoder}
%-------------------------------------------
% https://www.jianshu.com/p/ffd493e10751


%+++++++++++++++++++++++++++++++++++++++++++
\subsection{Auto-Encoder}

在说 VAE 之前,先来看一下它的前身 Auto-Encoder (AE)。AE 通过自监督的训练方式,
能够从原始特征获得一个潜在的特征编码,实现了自动化的特征工程,并且达到了降维和泛化的目的。
它的网络结构很简单,有编码和解码两个部分组成:

\begin{figure}[H]
\centering
\includegraphics[scale=0.3]{pix/dgm/encoder_decoder.png}
\caption{Auto-Encoder}
%\label{fig:label}
\end{figure}

容易看出,之所以是自监督就是因为网络的 target 即是 input 本身,因此不需要额外的标签工作,
虽然它由编码器和解码器两个部分组成,但是,显然从自编码器这个名字就可以看出,AE 的重点在于编码,
即得到这个隐藏层的向量,作为 input 的潜在特征,这是常见的一种 embedding 的一种方式。
而解码的结果,基于训练目标,如果损失足够小的话,将会与 input 相同,从这一点上看解码的值没有任何实际意义,
除了通过增加误差来补充平滑一些初始的零值或有些许用处。因为,从输入到输出的整个过程,
都是基于已有的训练数据的映射,尽管隐藏层的维度通常比输入层小很多,
但隐藏层的概率分布依然只取决于训练数据的分布,这就导致隐藏状态空间的分布并不是连续的,
于是如果我们随机生成隐藏层的状态,那么它经过解码将很可能不再具备输入特征的特点,
因此想通过解码器来生成数据就有点强模型所难了。


%+++++++++++++++++++++++++++++++++++++++++++
\subsection{VAE}

正是因为以上的这些原因,有大佬就对 AE 的隐藏层做了些改动,得到了 VAE。

\begin{figure}[H]
\centering
\includegraphics[scale=0.3]{pix/dgm/variational_encoder_decoder.png}
\caption{Variational Auto-Encoder}
%\label{fig:label}
\end{figure}

VAE 将经过神经网络编码后的隐藏层假设为一个标准的高斯分布,然后再从这个分布中采样一个特征,
再用这个特征进行解码,期望得到与原始输入相同的结果,损失和 AE 几乎一样,
只是增加编码推断分布与标准高斯分布的 KL 散度的正则项,显然增加这个正则项的目的就是防止模型退化成普通的 AE,
因为网络训练时为了尽量减小重构误差,必然使得方差逐渐被降到 0,这样便不再会有随机采样噪声,也就变成了普通的 AE。

没错,我们先抛开变分,它就是这么简单的一个假设... 仔细想一下,就会觉得妙不可言。

它妙就妙在它为每个输入 $x$, 生成了一个潜在概率分布 $p(z|x)$, 然后再从分布中进行随机采样,
从而得到了连续完整的潜在空间,解决了 AE 中无法用于生成的问题。
《论语》有言:“举一隅,不以三隅反,则不复也。” ,给我的启发就是看事物应该不能只看表面,
而应该了解其本质规律,从而可以灵活迁移到很多类似场景。聪明人学习当举一反三,那么聪明的神经网络,
自然也不能只会怼训练数据。如果我们把原始输入 $x$ 看作是一个表面特征,而其潜在特征便是表面经过抽象之后的类特征,
它将比表面特征更具备区分事物的能力,而 VAE 直接基于拟合了基于已知的潜在概率分布,
可以说是进一步的掌握了事物的本质。就拿一个人的某个行为来说,我们不能单纯看行为本身,
因为这个行为往往代表了他的综合特性,他从出生开始,因为遗传,后天教育和环境,决定了他在面对某种情形的情况下,
高概率会产生这个行为,也就是说掌握了概率分布,就掌控了一切。


Latent variable models form a rich class of probabilistic models that can infer hidden 
structure in the underlying data. In this post, we will study variational autoencoders, 
which are a powerful class of deep generative models with latent variables.


%+++++++++++++++++++++++++++++++++++++++++++
\subsection{Representation}

Consider a directed, latent variable model as shown below.

\begin{figure}[H]
\centering
\includegraphics[scale=0.618]{pix/dgm/vae.png}
\caption{Graphical model for a directed, latent variable model.}
%\label{fig:label}
\end{figure}

In the model above, $\bz$ and $\bx$ denote the latent and observed variables 
respectively. The joint distribution expressed by this model is given as
$$
p_\theta(\bx, \bz) = p(\bx \giv \bz)p(\bz).
$$

From a generative modeling perspective, this model describes a generative process for 
the observed data $\bx$ using the following procedure
\begin{align*}
\bz &\sim p(\bz) \\
\bx &\sim p(\bx \giv \bz).
\end{align}

If one adopts the belief that the latent variables $\bz$ somehow encode semantically 
meaningful information about $\bx$, it is natural to view this generative process as 
first generating the "high-level" semantic information about $\bx$ first before fully 
generating $\bx$. Such a perspective motivates generative models with rich latent 
variable structures such as hierarchical generative models $p(\bx, \bz_1, \ldots, \bz_m) 
= p(\bx \giv \bz_1)\prod_i p(\bz_i \giv \bz_{i+1})$---where information about $\bx$ 
is generated hierarchically---and temporal models such as the Hidden Markov Model---where 
temporally-related high-level information is generated first before constructing $\bx$.

We now consider a family of distributions $\mathcal{P}_\bz$ where $p(\bz) \in \mathcal{P}_\bz$ describes 
a probability distribution over $\bz$. Next, consider a family of conditional 
distributions $\mathcal{P}_{\bx\giv \bz}$ where $p_\theta(\bx \giv \bz) \in \mathcal{P}_{\bx\giv \bz}$ 
describes a conditional probability distribution over $\bx$ given $\bz$. Then our 
hypothesis class of generative models is the set of all possible combinations
\begin{align*}
\mathcal{P}_{\bx,\bz} = \set{p(\bx, \bz) \giv p(\bz) \in \mathcal{P}_\bz, p(\bx \giv \bz) \in 
\mathcal{P}_{\bx\giv\bz}}.
\end{align}

Given a dataset $\D = \set{\bx^{(1)}, \ldots, \bx^{(n)}}$, we are interested in the 
following learning and inference tasks
\begin{itemize}
%\setlength{\itemsep}{0pt}
%\setlength{\parsep}{0pt}
\setlength{\parskip}{0pt}
\item[-]
Selecting $p \in \mathcal{P}_{\bx,\bz}$ that "best" fits $\D$.

\item[-]
Given a sample $\bx$ and a model $p \in \mathcal{P}_{\bx,\bz}$, what is the posterior 
distribution over the latent variables $\bz$?

\item[-]
Approximate marginal inference of $\bx$: given partial access to certain dimensions 
of the vector $\bx$, how do we impute the missing parts?
\end{itemize}

We shall also assume the following
\begin{itemize}
%\setlength{\itemsep}{0pt}
%\setlength{\parsep}{0pt}
\setlength{\parskip}{0pt}
\item[-]
Intractability: computing the posterior probability $p(\bz \giv \bx)$ is intractable.

\item[-]
Big data: the dataset $\D$ is too large to fit in memory; we can only work with 
small, sub-sampled batches of $\D$.
\end{itemize}


%+++++++++++++++++++++++++++++++++++++++++++
\subsection{Learning Directed Latent Variable Models}

One way to measure how closely $p(\bx, \bz)$ fits the observed dataset $\D$ is to 
measure the Kullback-Leibler (KL) divergence between the data distribution (which 
we denote as $p_{\mathrm{data}}(\bx)$) and the model's marginal distribution 
$p(\bx) = \int p(\bx, \bz) \d \bz$. The distribution that ``best'' fits the data is 
thus obtained by minimizing the KL divergence. 
\begin{align*}
\min_{p \in \P_{\bx, \bz}}\KL{p_{\mathrm{data}}(\bx)}{p(\bx)}.
\end{align}

As we have seen previously, optimizing an empirical estimate of the KL divergence is 
equivalent to maximizing the marginal log-likelihood $\log p(\bx)$ over $\D$
\begin{align*}
\max_{p \in \P_{\bx, \bz}} \sum_{\bx \in \D} \log p(\bx) = \sum_{\bx \in \D} \log\int p(\bx, \bz) \d \bz.
\end{align}

However, it turns out this problem is generally intractable for high-dimensional $\bz$ 
as it involves an integration (or sums in the case $\bz$ is discrete) over all the 
possible latent sources of variation $\bz$. One option is to estimate the objective 
via Monte Carlo. For any given datapoint $\bf x$, we can obtain the following estimate 
for its marginal log-likelihood
$$
\log p(\bx) \approx \log \frac{1}{k} \sum_{i=1}^k p(\bx \vert \bz^{(i)}) \text{, where } \bz^{(i)} \sim p(\bz)
$$

In practice however, optimizing the above estimate suffers from high variance in 
gradient estimates. 

Rather than maximizing the log-likelihood directly, an alternate is to instead 
construct a lower bound that is more amenable to optimization. To do so, we note that 
evaluating the marginal likelihood $p(\bx)$ is at least as difficult as as evaluating 
the posterior $p(\bz \mid \bx)$ for any latent vector $\bz$ since by definition 
$p(\bz \mid \bx) = p(\bx, \bz) / p(\bx)$. 

Next, we introduce a variational family $\Q$ of distributions that approximate the 
true, but intractable posterior $p(\bz \mid \bx)$. Further henceforth, we will assume 
a parameteric setting where any distribution in the model family $\P_{\bx, \bz}$ is 
specified via a set of parameters $\theta \in \Theta$ and distributions in the 
variational family $\Q$ are specified via a set of parameters $\lambda \in \Lambda$. 

Given $\P_{\bx, \bz}$ and $\Q$, we note that the following relationships hold 
true\footnote{The first equality only holds if the support of $q$ includes that of 
$p$. If not, it is an inequality.} for any $\bx$ and all variational distributions 
$q_\lambda(\bz) \in \Q$
\begin{align*}
\log p_\theta(\bx) &= \log \int p_\theta(\bx, \bz) \d \bz \\
&= \log \int \frac{q_\lambda(\bz)}{q_\lambda(\bz)} p(\bx, \bz) \d \bz\\
&\ge\int q_\lambda(\bz) \log \frac{p_\theta(\bx, \bz)}{q_\lambda(\bz)} \d \bz \\
&= \Expect_{q_\lambda(\bz)} \left[\log \frac{p_\theta(\bx, \bz)}{q_\lambda(\bz)}\right] \\
&:=\ELBO(\bx; \theta, \lambda)
\end{align}
where we have used Jensen's inequality in the final step. The Evidence Lower Bound 
or ELBO in short admits a tractable unbiased Monte Carlo estimator
\begin{align*}
\frac{1}{k} \sum_{i=1}^k \log \frac{p_\theta(\bx, \bz^{(i)})}{q_\lambda(\bz^{(i)})} \text{, where } \bz^{(i)} \sim q_\lambda(\bz),
\end{align}
so long as it is easy to sample from and evaluate densities for $q_\lambda(\bz)$. 

Which variational distribution should we pick? Even though the above derivation holds 
for any choice of variational parameters $\lambda$, the tightness of the lower bound 
depends on the specific choice of $q$. 

\begin{figure}[H]
\centering
\includegraphics[scale=0.318]{pix/dgm/klgap.png}
\caption{Illustration for the KL divergence gap between the marginal log-likelihood 
\(\log p_\theta(\bx)\) for a point \(\bx\) and the corresponding ELBO for a 
single 1D-parameter variational distribution \(q_\lambda(\bx)\).}
%\label{fig:label}
\end{figure}

In particular, the gap between the original objective(marginal log-likelihood 
$\log p_\theta(\bx) $) and the ELBO equals the KL divergence between the approximate 
posterior $q(\bz)$ and the true posterior $p(\bz \giv \bx)$. The gap is zero when the 
variational distribution $q_\lambda(\bz)$ exactly matches $p_\theta(\bz \giv \bx)$. 

In summary, we can learn a latent variable model by maximizing the ELBO with respect 
to both the model parameters $\theta$ and the variational parameters $\lambda$ for any 
given datapoint $\bx$

\begin{align}
\max_{\theta} \sum_{\bx \in \D} \max_{\lambda} \Expect_{q_\lambda(\bz)} \left[\log \frac{p_\theta(\bx, \bz)}{q_\lambda(\bz)}\right].
\end{align}


%+++++++++++++++++++++++++++++++++++++++++++
\subsection{Black-Box Variational Inference}




%+++++++++++++++++++++++++++++++++++++++++++
\section{变分自编码是怎么炼成的}
%-------------------------------------------

读了上面的内容之后,你应该对 VAE 模型有了一个较为直观和感性的认知,但是可能会疑惑所谓的变分到底在哪里?
放心,变分并没有被作者吃掉,接下来,我们就从变分推断的角度,对 VAE 进行一个理性的推导。有了上面的基础,
再读下面的内容时就会轻松愉快很多。


%+++++++++++++++++++++++++++++++++++++++++++
\subsection{变分推断}

变分自编码器(VAE)的想法和名字的由来便是变分推断了,那么什么是变分推断呢?
变分推断是 MCMC 搞不定场景的一种替代算法,它考虑一个贝叶斯推断问题,给定观测变量 $x\in\mathbb{R}^n$ 
和潜变量 $z\in\mathbb{R}^d$, 其联合概率分布为 $p(z,x) = p(z)p(x|z)$, 目标是计算后验分布 
$p(z|x)$。然后我们可以假设一个变分分布 $q(z)$ 来自分布族 $Q$,通过最小化 KL 散度来近似后验分布 
$p(z|x)$:
$$
q^* = \operatorname{arg min}_{q(z)\in Q} D_{KL}(q(z) ~ \| ~ p(z|x))
$$
这么一来,就成功的将一个贝叶斯推断问题转化为了一个优化问题。


%+++++++++++++++++++++++++++++++++++++++++++
\subsection{变分推导过程}

有了变分推断的认知,我们再回过头去看一下 VAE 模型的整体框架。VAE 就是将 AE 
的编码和解码过程转化为了一个贝叶斯概率模型:我们的训练数据即为观测变量 $x$,
假设它由不能直接观测到的潜变量 $z$ 生成。于是,生成观测变量过程便是似然分布:$p(x | z)$,
也就是解码器,因而编码器自然就是后验分布:$p(z | x)$。根据贝叶斯公式,建立先验、后验和似然的关系:
$$
p(z|x) = \frac{p(x|z)p(z)}{p(x)} = \int_z\frac{p(x|z)p(z)}{p(x)}dz
$$
接下来,基于上面变分推断的思想,我们假设变分分布 $q_x(z)$, 通过最小化 KL 散度来近似后验分布 $p(z|x)$,
于是,最佳的 $q_x^*$ 便是:
\begin{align*}
q_x^* &= \operatorname{arg min}\left( D_{KL}(q_x(z) \| p(z|x)) \right) \\
&= \operatorname{arg min}\left( \mathbb{E}_{q_x(z)}\left[ 
\log q_x(z) - \log p(x|z) - \log p(z) \right] + \log p(x) \right)
\end{align}
因为训练数据 $x$ 是确定的,因此 $\log~p(x)$ 是一个常数,于是上面的优化问题等价于:
\begin{align*}
q_x^* &= \operatorname{arg min}\left( \mathbb{E}_{q_x(z)}\left[ 
\log q_x(z) - \log p(x|z) - \log p(z) \right] \right) \\
&= \operatorname{arg min}\left( \mathbb{E}_{q_x(z)}\left[ 
 - \log p(x|z) + (\log q_x(z) - \log p(z)) \right] \right) \\
&= \operatorname{arg min}\left( \mathbb{E}_{q_x(z)}\left[ 
 - \log p(x|z) + D_{KL}(q_x(z) ~\|~ p(z)) \right] \right)
\end{align}

此时,优观察一下优化方程的形式,已经是我们前面所说的 VAE 的损失函数了。
显然,跟我们希望解码准确的目标是一致的。要解码的准,则 $p(x|z)$ 应该尽可能的小,编码特征 $z$ 
的分布 $q_x(z)$ 同 $p(z)$ 尽可能的接近。此时恰好 $-\log p(x|z)$ 和 $D_{KL}(q_x(z) \| p(z|x))$ 
都尽可能的小,与损失的优化的目标也一致。


%+++++++++++++++++++++++++++++++++++++++++++
\subsection{如何计算极值}

正如前面所提到的 AE 潜变量的局限性,我们希望 VAE 的潜变量分布 $p(z)$ 应该能满足海量的输入数据 $x$ 
并且相互独立。基于中心极限定理,以及为了方便采样,我们有理由直接假设 $p(z)$ 是一个标准的高斯分布 
$\mathcal{N}(0,1)$。


\subsubsection{编码部分}

我们先来看一下编码部分,我们希望拟合一个分布 $q_x(z) = \mathcal{N}(\mu, \sigma)$ 尽可能接近 
$p(z) = \mathcal{N}(0,1)$。关键就在于基于输入 $x$ 计算 $\mu$ 和 $\sigma$。直接算有点困难,
于是就使用两个神经网络 $f(x)$ 和 $g(x)$ 来无脑拟合 $\mu$ 和 $\sigma$。

值得一提的是,很多地方实际使用的 $f(x)$、$g(x)$ 两部分神经网络并不是独立的,而是有一部分交集。
即他们都先通过一个 $h(x)$ 映射到一个中间层 $h$, 然后分别对 $h$ 计算 $f(h)$ 和 $g(h)$。
这样做的好处一方面是可以减少参数数量,另外也会导致拟合的效果差一些,算是防止过拟合吧。


\subsubsection{解码部分}

解码,即从潜变量 $z$ 生成数据 $x$ 的过程,在于最大化似然 $p(x|z)$,那这应该是个什么分布呢?
通常我们假设它是一个{\bf 伯努利分布}或是{\bf 高斯分布}。
这是因为伯努利分布简单,高斯分布自然。

知道了分布类型,那计算 $-\log p(x|z)$ 最小值其实只要把分布公式带进去算就可以了。

\begin{itemize}
%\setlength{\itemsep}{0pt}
%\setlength{\parsep}{0pt}
\setlength{\parskip}{0pt}
\item[-]
高斯分布
\begin{align*}
\operatorname{arg min}(-\log q(x|z))
&= \operatorname{arg min}\frac{1}{2}
\left\| \frac{x-\tilde{\mu}(z)}{\tilde{\sigma}(z)} \right\|^2 
+ \frac{c}{2}\log 2\pi + \frac{1}{2} \\
&= \operatorname{arg min}\frac{1}{2}
\left\| \frac{x-\tilde{\mu}(z)}{\tilde{\sigma}(z)} \right\|^2
\end{align}
和预期一样,演变为了均方误差。

\item[-]
伯努利分布
假设伯努利的二元分布是 $p$ 和 $1-p$ (注意这里是输出没一维组成的向量)
$$
\operatorname{arg min}(-\log q(x|z)) = \operatorname{arg min}(-x\log p - (1-x)\log(1-p))
$$
很对,正好就是交叉熵的损失。
\end{itemize}

然后,将编码和解码部分组合到一起,就形成了完整的 VAE 网络。



%+++++++++++++++++++++++++++++++++++++++++++
\subsection{一点微不足道的技巧}

训练的时候似乎出了点问题。从编码得到的分布 $\mathcal{N}(\mu,\sigma)$ 随机采样 $z$ 
的这个过程没法求导,没法进行误差反向传播。这里可以使用一个叫做重参数(reparametrisation trick) 的技巧:
$$
z=f(x)\zeta + g(x), \qquad \zeta\sim N(0,1)
$$
真是妙呀,这样一来将采样变成了一个数值变换,整个过程便可导了。

这样,训练好模型之后,我们可以直接将解码部分拿出来,通过标准高斯分布随机采样源源不断的生成数据了。

%+++++++++++++++++++++++++++++++++++++++++++
\subsection{延伸的一些思考}

VAE 中使用神经网络来拟合高斯分布的想法独树一帜,对很多任务能带来启发。
神经网络在特征化训练数据的过程无异于海量数据的管中窥豹,但是想要让模型超脱于豹,
而像人一样产生对相似的猫、老虎...之类的概念,VAE 的这种思想颇有一些意味。

\href{https://arxiv.org/pdf/1312.6114.pdf}{Auto-Encoding Variational Bayes}

\href{}{Understanding Variational Autoencoders}

